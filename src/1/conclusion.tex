\chapter{Conclusion}

Les décharges électrostatiques sont une source majeure de stress pour les composants électroniques.
Ce stress peut se traduire par une casse matérielle, et ce type de problème est le plus largement étudié dans le domaine des ESD.
Récemment, une nouvelle catégorie de défaillances est apparue appelée défaillance fonctionnelle.
Elles se traduisent par une perturbation temporaire des systèmes électroniques à la suite d'une décharge.
C'est d'autant plus problématique que de nos jours les systèmes électroniques ont des responsabilités accrues vis à vis de notre sécurité.

% Chap 1
Dans le chapitre 1, un état de l'art sur les moyen de tests ESD a été présenté.
Il permet d'identifier les sources classiques de perturbation.
Puis, des outils d'observation et des méthodes de modélisation de fautes ont été détaillés.

% Chap 2
Le chapitre 2 est focalisé sur une méthode de modélisation des systèmes électroniques pour l'ESD.
Une approche modulaire a été favorisée, afin d'avoir des composants réutilisables en vues de modéliser des systèmes complexes.
Un générateur TLP a été modélisé en utilisant cette approche.
% TLP-HMM
Puis, l'expérience acquise via cette étude a permis de développer un nouveau type de générateur HMM basé sur un TLP.
Il offre plusieurs avantages par rapport à un générateur classique, mais requiert encore de multiples améliorations.
% Near-field current processing
Enfin, une méthode de traitement des sondes de champ proche sur puce a été présentée.
Le programme de traitement des données a été libéré \cite{} sous une licence open-source afin de favoriser des améliorations futures.
Deux méthodes différentes furent proposées pour reconstituer la forme originale du courant.

% Chap 3
Le chapitre 3 présente un cas d'étude de défaillance, où une fonction de régulation redémarre à cause d'un ESD.
Le cas de défaillance est étudié, puis un véhicule de test est développé afin d'acquérir plus de données.
Il contient la fonction analogique en question, ainsi que de multiples structures dédiées à sa surveillance et à des mesures.
Malheureusement, des problèmes liés au système de communication du véhicule de test ont empêché d'exploité correctement cette puce.
Des corrections futures devraient permettent de résoudre ce problème.

% Chapter 4
Enfin, le chapitre 4 est focalisé sur le développement d'outils de simulation pour la prédiction de défauts fonctionnels.
Les outils de simulation permettent d'identifier des problèmes très tôt dans le flot de conception, et sont donc extrêmement intéressants.
La principale difficulté pour les ESD au niveau circuit-intégré est la complexité des circuits.
Les problèmes de convergence sont très courants et parfois difficiles à éviter.
La complexité des circuits rends l'investigation assez laborieuse.

% First modelling method
La première méthode est à destination des concepteurs de circuits-intégrés.
Elle consiste à caractériser individuellement en simulation des blocs dans la puce.
La caractérisation est effectuée entre une broche d'entrée et de sortie, à l'aide de signaux impulsionnels rectangulaires.
Puis, ces modèles comportementaux sont connectés ensemble et permettent de propager une perturbation, pour déterminer ses variations globales en terme d'amplitude et de durée.
Cette approche a été testé avec de bons résultats sur le même cas d'étude que celui fabriqué sur silicium.
Il y a de nombreuses améliorations à apporter, mais cette méthode semble d'ores et déjà prometteuse.

% Second modelling approach
La seconde méthode est à destination des équipementiers.
Elle cherche à construire un modèle boite-noire des circuits intégrés, entre une broche d'entrée externe et de sortie externe.
Une fonction de la puce est caractérisée à l'aide encore une fois de signaux rectangulaires impulsionnels, et un modèle boite-noire en est déduit.
Encore à ses débuts, la méthode présente des problèmes pour modéliser des broches fournissant un courant en statique.
Des améliorations futures sont requises pour corriger ces problèmes.

% Final Conclusion
En conclusion, ce travail a permis d'explorer différentes pistes dans ce champ de recherche relativement récent qu'est l'analyse et la prédiction de faiblesses fonctionnelles pour l'ESD.
L'objectif était de fournir des méthodes et outils réutilisables, plutôt que de se focaliser sur un cas d'étude trop spécifique.
La plupart des méthodes présentées ici restent au stade de prototype, et demandent des améliorations futures, mais peuvent servir de base de travail pour d'autres travaux de recherche.
